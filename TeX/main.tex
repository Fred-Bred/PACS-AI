\documentclass[12pt]{report}
\usepackage{cite}
\usepackage{amsmath,amssymb,amsfonts}
\usepackage{algorithmic}
\usepackage{graphicx}
\usepackage{textcomp}
\usepackage{xcolor}
\usepackage{booktabs}
\usepackage{tabularx}
\usepackage{hyperref}
\usepackage{setspace}

%KUstyle
% \usepackage{KUstyle}
% \ptype{Social Data Science}
% \subtitle{An Attachment-Based Therapy Metric}


% Citation style packages
% \usepackage{apacite} % Package for APA citations
% \bibliographystyle{apacite}
\bibliographystyle{plain}

\onehalfspacing

% This changes the content of the frontpage
\title{Anxious or Avoidant? Securing Reliable Repeated Measures of Adult Attachment Through Machine Learning}
\author{Frederik Bredgaard}
\date{May 31st 2024}

\renewcommand{\contentsname}{Table of Content}

\begin{document}

\maketitle
\
\tableofcontents

\chapter*{Introduction}
The framework of attachment theory has continuously delivered new and valuable insights into the development of children and the relationships, pathologies, and treatments of adults through the last six decades since its inception.
However, its clinical application has been limited, arguably not for a lack of clinical relevance but rather due, in large part, to the cumbersome measurement of the constructs belonging to the theory.
As will be covered below, several instruments and methods for the assessment of attachment style exists. However, the gold standard for assessing adult attachment, the Adult Attachment Interview (AAI) \cite{AAITest}, is a time-consuming measure, making its application in both large-scale research and clinical settings rather limited.
The objective of this thesis is to develop on existing methods, specifically the Patient Attachment Coding System \cite{Talia2017}, adding a degree of automation through language modelling approaches derived from machine learning.
While the available data is limited at this stage of the field, I believe that future work can build on the approach developed here to produce statistical models for autocoding the AAI as well as a clinically relevant tool that can assist clinicians and researchers alike by making the assessment of attachment in adults more easily available and significantly more scalable.
As such, the present thesis will investigate the following research question:
\begin{quote}
    Can automatic natural language processing be applied to assess patients' attachment characteristics in ways meaningful for research and clinical applications in psychotherapy?
\end{quote}
To do this, I will investigate the feasibility and utility of automatically assessing psychotherapy patients' attachment characteristics.
The clinical utility of the approach is assessed first, through a review of the link between theory and empirical data, including the implications of attachment style on health, happiness, and well-being.

Second, the feasibility of automatically classifying a patient's attachment style is investigated through a series of language modelling experiments, based on the limited available data.
% Add something on the discussion here.
\chapter{Attachment: Background and Theory}
This section outlines the most central elements of attachment theory from its development to the clinical relevance of different patterns of attachment and how the theory more broadly explains or mediates the effects of psychotherapy.

From this theoretical perspective, an attachment may be understood as an affectional tie formed between an individual and some other and which is characterised by behaviours seeking to gain and maintain proximity to the other.
This tie and its associated behaviours bind the two individuals together in space and endures over time with the proximity-seeking behaviours being particularly prevalent during times of distress and varying in what is referred to as attachment patterns or styles \cite{Ainsworth1970,Bowlby1988}.
In the following, the historical development of this understanding and the empirical frameworks originating from and supporting it are covered first, before a review of relevant literature is presented.

\section{Theoretical Development}
The theory of attachment is a fundamentally ethological approach, which at the time of its development sought to explain behaviours that were poorly accounted for in existing theories.
The initial development of the theory is often credited to John Bowlby \cite{Bowlby1969attachment,Bowlby1973separation,Bowlby1980loss}, who originally formulated it from a psychoanalytic, Freudian perspective.
However, the framework would soon develop through a more empirically informed approach, not least thanks to the groundbreaking work on infant attachment done by Mary Ainsworth and her colleagues \cite{Ainsworth1970}.

Bowlby's arrival at the tenets of attachment theory was influenced by the psychoanalytic exploration of object relations - a particular area of Freudian thought concerned with ego development and the relation of the psyche to external objects and persons.
Bowlby, however, viewed the apparent attachment behaviours as being mostly adaptive to external needs and stimuli rather than as responses to internal fantasies. In line with the tradition that inspired him, Bowlby based his initial theories on case studies, particularly of delinquent children (e.g. \cite{bowlby1946thieves}). From these cases, Bowlby and his colleagues found that children displaying criminal or otherwise problematic behaviours or mental or emotional distress typically struggled to form meaningful relationships and many had been repeatedly institutionalised or moved between foster homes \cite{bowlby1951WHO}.

While the foundation for the theory and its conclusions had been laid long before, the first widely influential published piece on the effects of childhood care and attachment on mental health was Bowlby's 1951 monograph published by the World Health Organization under the title \textit{Maternal Care and Mental Health} \cite{bowlby1951WHO}.
The central assertion of this work, which reviewed the limited existing empirical data, was that healthy infant development presumes a warm and continuous relationship with the mother.
At the time, this proposition was controversial as it broke with traditional views on child-rearing and development. The dominant view at the time considered physical contact with infants to be harmful, leading nurseries and hospitals to be sterile and contact-less \cite{Karen1994}.
Bowlby's view contrasted with this by emphasising the relational and emotional needs of children. As such, Bowlby found himself opposing the psychoanalytic views that originally inspired him and that attributed the infant's bond with their mother to internalized needs and unconscious sexual desires.
At the time, psychoanalysts were mainly opposed in their views on development by behaviourist perspectives that viewed the same bond as a result of learned feeding behaviours.

However, neither psychology nor medicine offered any well-developed coherent theoretical explanations for the mechanisms leading from maternal care to mental health \cite{Bowlby1988, who1962deprivation} and Bowlby did not provide thorough theoretical explanations for the infant-mother bond in this publication.
Nevertheless, the publication successfully brought attention to the importance of maternal care for developing children.

Following the critique of his 1951 monograph, the search for theoretical mechanisms and answers to the yet unexplained observations published in his monograph led Bowlby to ethology.
Here, in the study of animal behaviour, he found the concepts that would, in combination with cybernetics and psychoanalysis, eventually form the building blocks of his own theory, namely the concepts of critical periods or imprinting, instincts, and behaviour control systems \cite{Bowlby1988,bowlby1953critical}.

By 1958 Bowlby's interdisciplinary efforts converged in a rare intersection of co-occurring empirical and theoretical advancements. The theoretical advancement came as Bowlby published \textit{The Nature of the Child's Tie to his Mother} \cite{Bowlby1958} developing the concept of attachment behaviours and advancing his view that the bond between infants and their mothers developed thanks to the infant's innate need for closeness.
The same year, inspired by discussions with Bowlby, Harry Harlow published an account of a series of experiments with rhesus macaques and artificial surrogate mothers titled \textit{The Nature of Love} \cite{Harlow1958}.
In these famous experiments, Harlow constructed surrogate mothers from metal wire and planks of wood and raised the macaques with only these surrogates as caretakers. The various experiments determined that infants would become attached to their particular mother, easily recognising and preferring it to all others.
Three findings in particular would lay the foundation to the developing understanding of attachment.

First, Harlow investigated the preference of infant macaques between soft, clothed surrogate mothers and bare wire mothers who provided nourishment through bottle-feeding.
He learned that the macaque infants preferred the comfort provided by the mothers covered in a soft cloth, regardless of with which mother he placed the feeding bottle.
In conditions where the bare-wire mother held the bottle, infants would visit her only briefly to feed before returning to the comfort of the cloth mother.
This seems to indicate that the infant-mother bond is founded on desires separate from the need to feed.

Second, Harlow found, in experiments that included open fields to be explored and frightening objects making loud sounds to intimidate the macaque infants that the mothers functioned as a secure base for exploration.
This was apparent by the infants fleeing to seek the comfort and protection of their cloth mothers upon encountering novel and frightening situations and by their lack of exploration when no mother was present.
These observations became central to the understanding of attachment figures as bases for exploration, enabling healthy infant development even in humans.

Finally, Harlow noticed that, compared to infants raised by cloth mothers, those raised by wire mothers had frequent digestive issues such as diarrhoea. Harlow attributed this to the stress induced by insufficient comfort provided by the wire mothers, providing the first empirically founded theoretical link between attachment deprivation and mental health \cite{Harlow1958}.

Through the next two decades Bowlby would elaborate on the theory of attachment over several major publications, particularly the three-volume magnum opus \textit{Attachment and Loss} \cite{Bowlby1969attachment, Bowlby1973separation,Bowlby1980loss} released between 1969 and 1980.
In 1988, almost four decades post its initial release, he claimed to have finally remedied the empirical and theoretical deficiencies of \textit{Maternal care and mental health} \cite{Bowlby1988}.

Inspired by the traditions of ethology, cybernetics, and cognitive psychology, Bowlby considered the development of attachment and its related behaviours from both ontogenetic and phylogenetic perspectives.
In the phylogenetic view, an attachment system developed through human, and generally mammalian, evolutionary history to encourage infants to maintain proximity to their caretakers, particularly so during conditions of threat.
Bowlby theorised that this system, conceptualised as a fusion of ethology and cybernetics, evolved to encourage behaviours that regulate affect, fear, security, and exploration to achieve homeostasis.
The attachment system is activated when a threat is perceived and attachment behaviours tend to follow. The nature of these behaviours depends on the attachment style of the individual which determine the strategies employed to regulate affect and cope with distress and the perceived threat.
The individual's attachment style is developed ontogenetically through the interplay of innate tendencies -- or instincts -- and interactions with caregivers. In other words, infants develop expectations of the responsiveness of their caregivers based on experience.
These expectations tend to stay with them into adolescence and adulthood, forming rather stable mental models, a concept borrowed from cognitive psychology, of close and intimate relationships which influence expectations, behaviours, and self-regulation throughout the life-span.

\subsection{Assessing Attachment}
A large part of the move towards empirical methods and readily applicable theory constructs is attributable to the work of Mary Ainsworth whose contributions greatly nuanced the understanding of attachment and maternal deprivation.
She introduced the classification of attachment patterns by observing the proximity-seeking behaviours of children as they were reunited with their mothers in a strange situation  \cite{Ainsworth1979, Ainsworth1970}.
Ainsworth originally proposed three different patterns of behaviour based on her observations of infants.
In a 1970 study of infants aged 49-51 weeks, Ainsworth and Bell studied attachment and exploration behaviours during a procedure called \textit{The Strange Situation} in which infants are introduced to an unfamiliar environment with their mothers and briefly left alone with a stranger.
Ainsworth and Bell found that upon reunion with their mothers, infants' behaviours could be categorised into three broad classes.

The first group of infants, classified as secure, felt increasingly safe to explore the experimental room with their mother present, treating her as a secure base for exploration to which they could return to have their attachment needs met.
During separation episodes, these infants showed less exploration and increased attachment behaviours such as crying and looking around for the mother. Upon reunion, infants later classified as secure increased proximity-seeking and contact-maintaining behaviours.

However, for a subset of infants, both contact-maintaining \textit{and} contact-resisting behaviours were heightened in the reunion episodes.
This suggests that these infants have great need for proximity but may feel anger towards the mother following the separation episode, which is generally interpreted as ambivalence, thus naming the ambivalent childhood attachment classification.

Finally, some infants appeared almost indifferent to the separation and displayed proximity-avoiding behaviours during reunion episodes.
These infants would be classified as avoidant.

\subsubsection{Adult Assessment Tools}
The theory was expanded into a clinically applicable framework for adults when the Adult Attachment Interview (AAI; \cite{AAITest}) was developed to assess the adult equivalents of the childhood patterns of attachment behaviours.
The AAI is a semi-structured interview that asks respondents about autobiographical memories from early childhood related to attachment.
The interview is coded according to the ways in which these childhood experiences are described and the reflections offered by respondents.
Thus, the coding process is less concerned with the valence or specific content of memories and more interested in the organisation of thoughts and memories.
The AAI classifies individuals as secure if they provide coherent and mostly complete descriptions of their childhood experiences.
The childhood classification for ambivalent attachment is replaced with the preoccupied pattern.
Individuals with this classification provide incoherent, vague, or emotionally exaggerated narratives of their childhood experiences.
Likewise, the childhood classification of avoidant attachment is replaced in adults with the dismissing classification.
Individuals with this classification, tend to provide overly succinct, somewhat superficial, and unemotional accounts of their childhood experiences \cite{Hesse1999, AAITest}.

Perhaps owing to the immense empirical support and validation for the AAI \cite{BakermansKranenburg1993, IJzendoorn1995, Crowell1996Discriminant, Sagi1994}, it has become the de facto gold standard of attachment research \cite{AAITest, Talia2019, haltigan2014adult}.
The administration of the AAI takes between 60 and 90 minutes while the required verbatim transcription may take four to ten hours and coding of the transcript is expected to take at least four hours by a coder requiring extensive training, prompting some researcher to search for simpler or less resource-intensive methods of measurement \cite{Haas1994}.

Most commonly, researchers have attempted to develop less demanding measures for use in academic research.
This has resulted mostly in questionnaire-based measures (e.g., the Experiences in Close Relationships Scale, \cite{Brennan1998}; the Relationships Questionnaire, \cite{Bartholomew1991}; Attachment Style Questionnaire, \cite{Feeney1994ASQ}; Reciprocal Attachment Questionnaire, \cite{West1992}) but also interviews (e.g., Current Relationship Interview, \cite{Crowell1996}) and a picture-coding system \cite{George2012}.
While these alternatives offer more scalable application for large-scale research and efficient measurement for clinical practice, they do so at the cost of at least some reliability and validity. Namely, Compared to the AAI, self-report measures tend to show lower retest reliability \cite{Scharfe1994}.
Longer self-report measures appear to be more reliable \cite{Sibley2005,Wongpakaran2021}, but the efforts to validate most measures has been rather limited.

Nevertheless, I would argue that the most paradigm-shifting developments in methods of measurement have only come within the last decade.
This development consists of indirect measures in which transcripts are taken of psychotherapy sessions and discursive or narrative patterns are analysed for indicators of attachment characteristics.
In this area, Slade \cite{Slade2016} highlights the Patient Attachment Coding System (PACS) as a groundbreaking method for assessing in-session attachment dynamics and inferring more stable patient characteristics.
Across modalities, experienced clinicians can attend to attachment dynamics in their psychotherapy sessions.
As such, the analysis of in-session behaviour to infer attachment organisation may not be new, but the formalisation offered by Talia, Miller-Bottome, and Daniel \cite{Talia2017, Talia2014} provides a testable, valid measurement. Further, in this thesis, the approach provides the basis for the automation of attachment assessments from in-session behaviour in terms of both theory and data.
The PACS analyses the speech of patients in psychotherapy and studies how the patient manages emotional proximity with the therapist by eliciting, maintaining, or avoiding emotional attunement from the therapist.
The PACS takes notice of these behaviours by coding 59 related discursive markers making up 12 subscales, which in turn make up five main scales.
The five main scales assessed are Proximity Seeking, Contact Maintaining, Exploring, Avoidance, and Resistance.
The scores on these five scales can be used to infer the patient's classification on the AAI through a simple algorithm.
Specifically, a patient's in-session attachment is classified as secure if scores are highest on Proximity Seeking, Contact Maintaining, or Exploring. Avoidant or preoccupied classifications are assigned if the scales Avoidance or Resistance, respectively, are higher.
In a 2017 validation study, Talia et al. \cite{Talia2017} found excellent inter-rater agreement for three-way classification with independent coders using the PACS ($r=.91, \kappa = .87, p<0.001$).
Comparing the PACS and AAI classifications for the same patients, the authors found $r=0.87$ $(\kappa = .81, p<0.001)$, also indicating excellent agreement.

In summary, the PACS provides a novel method for assessing attachment on the basis of in-session behaviour, showing what is according to convention \cite{Cicchetti1994} both excellent reliability and excellent concurrent validity with the Adult Attachment Interview.

\section{Current View of Attachment and Its Significance for Health and Daily Life}
Whereas most early theory statements and empirical analyses focused on the mother-child relationship, there is now a vast literature extending the theory to other caregivers and relationships more broadly including fathers, extended family members, educators, friends, and romantic relationships. For adults, attachment style is generally considered relevant to almost all relations, ranging from close family and romantic partners to friends and professional relationships and Bowlby's original conception of an attachment hierarchy \cite{Bowlby1969attachment} has been expanded to include almost any relation imaginable (see e.g. \cite{Rosenthal2010, Karantzas2011ArthritisAS, Rowe2005, Tancredy2006} for examples of research using rankings or comparisons of attachment figures).

After a summary of the current view of central constructs, the following sections offer a non-exhaustive review of relevant links between theory, practice, and observable outcomes, relying on the understanding of attachment styles measured using different approaches as essentially analogous or interchangeable even if they are presented with different names, as is custom in meta-analyses combining data from different measures (e.g., \cite{McConnell2011,Pinquart2013}).

\subsection{Current Understanding and Central Constructs}
\subsubsection*{Attachment Figures}
Bowlby \cite{Bowlby1969attachment} identified three core criteria that define an attachment figure.
The first involves serving as a target of proximity seeking and maintenance behaviours.
Throughout life, individuals naturally gravitate towards their attachment figures for feelings of safety and comfort.
Conversely, separation from these figures can be a source of distress.
Attachment behaviours are increased when a threat is perceived or in times of distress, through the activation of the attachment system.
Even in the absence of overt attachment behaviours, the attachment system is likely never fully deactivated \cite{Bowlby1969attachment}.
Secondly, an attachment figure must function as a safe haven. This entails providing a secure refuge that facilitates emotional regulation.
Through support, soothing, and comfort, attachment figures help individuals manage their emotions effectively and eventually deactivate their attachment system.
Finally, the safe haven provided by an attachment figure also serves as a secure base for exploration.
Following a distressing event, the safety of the attachment figure helps down-regulate the attachment system, allowing for exploration to continue \cite{Bowlby1969attachment}.
This exploration can encompass the physical world, oneself, relationships with others, and even one's own emotions.
Crucially, the sense of security offered by the attachment figure allows individuals to develop their own capacities and personality.
According to the attachment-based model of therapeutic change, it is exactly the fulfilment of these three provisions that enables therapeutic progress not only in changing attachment styles but in all matters \cite{Bowlby1969attachment, Mikulincer2013}.

\subsubsection*{Scales and Classifications}
Attachment is measured with a variety of methods with varying degrees of granularity. Despite the methodological differences, these methods converge on a common understanding of the underlying structure of attachment based on the continuous orthogonal dimensions of anxiety and avoidance, allowing the classification into distinct styles \cite{Brennan1998,Mikulincer2013}.
This structure reflects the main tenets of attachment theory as it is understood and applied today and a person's location on these two dimensions corresponds to their overall attachment security, reflecting the way in which they will interpret and react to stressors \cite{Mikulincer2007}.

The anxiety scale corresponds to some degree to one's perception or mental model of oneself. Attachment anxiety is ultimately the self-doubt and worry related to a lack of confidence that attachment figures will meet one's attachment needs.
When relationship ruptures occur or insecurity appears, individuals with high attachment anxiety will tend to overcompensate with exaggerated attachment behaviours.
The avoidance scale is roughly analogous to one's mental models of others. Attachment avoidance can be reduced to a lack of trust in the kindness of potential attachment figures, leading one to avoid them and potentially forego deep emotional connection altogether.
This understanding of attachment as existing in two orthogonal dimensions is the one most widely accepted today.
Nevertheless, the classifications remain widely used in both scientific studies and clinical applications \cite{Mikulincer2013}.

\subsubsection*{Attachment Styles}
An attachment style or pattern is a pattern in behaviour, thought, mental models, and emotional reactions relating to close and intimate relationships, roughly defined by where an individual lies in the two-dimensional space formed by avoidance and anxiety.

Secure attachment, corresponding to low scores on both dimensions, is associated with the belief that one is worthy of love and close relationships and that others are capable of providing for one's attachment needs and are generally reliable and trustworthy.
This roughly corresponds to holding positive mental representations of both the self and others. This means that secure individuals tend to be more comfortable in both expressing emotions and seeking support and more effective and constructive in their affect-regulation.

Importantly, insecure attachment is often experienced as either a lack of or an excess of emotions and can be expressed through different coping mechanisms which also tend to form somewhat predictable patterns.
This tends to manifest as either preoccupied or dismissing patterns of attachment. Preoccupied patterns of attachment, described by high anxiety and low avoidance, is characterised by the belief that one is less worthy of love and close relationships and anxiety around the certainty of others and their relationships.
This may lead preoccupied individuals to require more reassurance and to struggle with emotion regulation, potentially becoming somewhat dependent on their attachment figures for regulation and support \cite{Hudson2020}.

In contrast, the dismissing attachment pattern, described by low anxiety and high avoidance, is characterised by discomfort with intimacy and closeness brought on by the belief that others are unreliable and cannot be trusted to support one's attachment needs and emotion regulation.
To cope, dismissing individuals may suppress their emotions and even come to view emotions as weakness.
This leads dismissing individuals to avoid close relationships and prioritise self-reliance and independence \cite{Mikulincer2013,Hudson2020}.

Individuals with fearful attachment, characterised by high levels of anxiety and avoidance, exhibit an emotional paradox in their deep desire for close relationships coupled with a fear of intimacy, leading to avoidance behaviours that cause significant distress \cite{Bartholomew1991}.

\subsubsection*{Coping: Hyper-Activating and Deactivating Strategies}
The experience of attachment insecurity and interpersonal problems is not unique to those with insecure attachment.
However, it is commonly suggested that attachment insecurity manifests through two primary strategies to deal with insecurity, uncertainty, and ruptures in relationships: hyper-activation or deactivation of the attachment system \cite{Mikulincer2003, Mikulincer2013,Tyrrell1999, Slade2016}.
This conceptualisation attempts to more broadly explain the emotions and behaviours associated with attachment-related events or perceptions that such events have or are occurring, by highlighting the coping strategies employed by people according to their attachment classification.
Namely, people who score high on attachment anxiety tend to reach for hyper-activating strategies.
These consist of bursts of intense attachment behaviours, the aim of which is to achieve support and love in spite of the individual's lack of confidence that these things will be provided.
Conversely, deactivating strategies are employed mainly by individuals who score high on the avoidance scale.
These strategies aim to down-regulate the attachment system by dismissing the importance of attachments and denying vulnerability and the need for closeness, thus minimising the potential distress experienced in case of rejection from or unavailability of attachment figures \cite{Mikulincer2003}.
Understanding the tendency to use either of these groupings of strategies is beneficial not only for individuals to understand themselves and their patterns of thought and behaviour, but, as I demonstrate in section \ref{sec:Applying attachment in therapy}, for therapists in tailoring their approach to building the therapeutic alliance and achieving favourable therapeutic outcomes.

\subsection{Stability and Change}
To understand predictive ability of attachment style, which constitutes the most direct link between theory and evidence, the assumption of stability must be addressed. Generally, an individual's state of mind with regard to attachment is theorised to be relatively stable throughout the lifespan. This theoretical assumption is based primarily on Bowlby's own speculation. Namely, in his seminal work \textit{Attachment and Loss}, first published in 1969, Bowlby theorises that expectations regarding attachment figures are acquired in infancy and early childhood and thus should remain stable throughout adulthood \cite{Bowlby1969attachment, Bowlby1973separation,Bowlby1980loss}. However, despite the stability of attachment styles being an axiom underpinning the theory as a whole, it has proven persistently difficult to confirm or reject the assertion empirically.

In the empirical literature, attachment classifications tend to show both stability and fluidity, with studies involving younger individuals finding only somewhat stable patterns with a greater degree of fluidity and studies of older individuals finding much greater stability.

Broadly, the available evidence points to mostly stable patterns of attachment over long periods of time. This is perhaps most convincingly demonstrated by a 2013 meta-analysis of 127 studies on attachment stability with a total \textit{N} of 21,072 by Pinquart, Fueßner, and Ahnert \cite{Pinquart2013}.
Here, the authors found medium-sized stability with a test-retest coefficient of .39. However, while no significant stability was detectable in studies running longer than 15 years, coefficients were larger for shorter time intervals and for samples not comprised of at-risk children.
This suggests that the applied attachment measures have good reliability, and that while attachment displays good stability for shorter time intervals, life events, circumstances, and perhaps deliberate interventions can change it over longer intervals.

Similarly, in a study by Zhang and Labouvie-Vief \cite{Zhang2004}, a sample of 370 individuals ranging in age form 15 to 87 was assessed regarding attachment, well-being, coping strategies, and depressive symptoms three times over a six-year period. Attachment style was found to be reasonably stable although changes were observed.
Test-rest reliability in this sample was between .40 and .49 between the first and second assessment and between .24 and .45 between the first and third measurement. This could indicate that attachment style does change, although this change is most likely to occur over a period of more than two years.
For comparison, Zhang and Labouvie-Vief cite studies demonstrating that test-retest reliability of big five personality traits over a similar six-year period tend to be around .60 - .80 after age 30 \cite{Costa1988,Roberts2000}.

The changes observed are themselves relevant to our understanding of the development, stability, and significance of attachment. Namely, Zhang and Labouvie-Vief found that change towards greater attachment insecurity was associated with defensive coping strategies characterised by rigid, immature and maladaptive ways of interacting with the world. Depressive symptoms was also a significant predictor of change towards greater insecurity. In contrast, change towards greater security was predicted by flexible and reality-oriented coping strategies and better perceived well-being.
Finally, Zhang and Labouvie-Vief found an effect of age on the direction of the observed changes. Specifically, it appears that over time, adults may become more secure and more dismissive but less preoccupied  \cite{Zhang2004}.

Lending further support to the notion of increased attachment stability with age, Consedine and Magai assessed attachment in 415 older adults at age 72 and again at age 78. In this sample, more than 80 \% of participants remained stable in their classification.

Studies into the hierarchies of attachment also further our understanding of the quality of the development of attachment through the lifespan.
Based on a study of mid- and late-adolescents recruited from a high school and a university, respectively, Rowe and Carnelley \cite{Rowe2005} concluded that peers become increasingly important attachment figures as people age. While the undergraduate students assessed did not rate their parents as any less important to their core self than the highschoolers did, they considered their friends significantly more central to their core self.
This and other studies (e.g. \cite{Fraley1997,Doherty2004}) paint the picture of adolescents expanding their circle of close attachments to include their peers, but that this expansion does not come at the expense of the strength of attachment to parents or primary caregivers.
As adolescents become adults, they tend to show a decline in how highly they rate their parents as attachment figures and how much they rely on them for attachment-related needs. In their place, there is strong support for the notion that adults tend to rely more on friends and romantic partners as they age and in particular as their peer- and romantic relationships go longer \cite{Tancredy2006,Doherty2004,Fraley1997}.

Attachment theory would lead one to believe that changes to attachment style are common following major life events such as loss, change in relationship status, or becoming a parent.
However, the specific causes of changes to attachment style are empirically not well understood.
While Kirkpatrick and Hazan \cite{Kirkpatrick1994} found relationship initiation or break-up to be a mediator of attachment change, many studies are unable to attribute any observed changes directly to life events.
One such study is an 8-month examination of 144 young adults by Scharfe and Bartholomew \cite{Scharfe1994}.
They used several forms of assessment and overall found moderate stability. The observed stability was noticeably higher for expert ratings than self-report measures, but, independent of measurement method, Scharfe and Bartholomew found no consistent relationship between changes in attachment security and life events between the two measurement occasions.
Similarly, Cozzarelli et al. \cite{Cozzarelli2003} did not find strong associations between a long list of life events and attachment changes over a two-year period in a sample of 442 women who underwent an abortion.

The best evidence on how attachment changes has come more recently. In a 2021 study, Fraley, Gillath, and Deboeck followed a sample of over 4,000 people assessed in multiple waves for between six and 40 months. They found that changes in attachment security occurred following life-events related to relationships, career, family, and more. However, most changes were transient and the majority of people would revert to their original attachment style given enough time after most life events.
Nevertheless, some events did tend to lead to enduring changes, suggesting that some experiences are likely to cause lasting changes to attachment style.
The greatest enduring effects on increasing general attachment anxiety occurred following such events as entering retirement, receiving a work-related promotion, starting school or university, or moving to a new location. A shared characteristic of these events is a great change in social network. Following such life-changing events, individuals likely enter new communities, perhaps leaving others, and an increased anxiety around relationships may follow from spending more time with new, less close, relationships which inherently feel less secure.
Simultaneously, spending less time in one's existing close relationships may weaken these relations or one's certainty of them, effectively making them less secure.
Notably, Fraley et al. only found one event which had a significant lasting effect on \textit{decreasing} general attachment anxiety. This event was finding out that oneself or one's partner was pregnant, which had the second-largest enduring effect on general attachment anxiety, only exceeded in magnitude by the opposite effect of retiring from work.
Compared to the anxiety measure, avoidance was more stable in this sample. In fact, Fraley et al. only found two events to be associated with significant enduring changes in general attachment avoidance. These were getting married, which was associated with increased avoidance, and oneself or one's partner giving birth, which was associated with decreased avoidance.
Importantly, there were significant individual differences in the extent to which people changed. In line with previous research (e.g. \cite{Zhang2004}), positive or negative appraisals of the given experience were related to the extent of attachment changes on the individual level.

Finally, the concept of volitional change to one's attachment style is important, yet poorly described empirically. Similarly to the research on the impact of life events, we have only recently seen strong direct evidence of deliberate changes to attachment security. This came in 2020 when Hudson, Chopik, and Briley \cite{Hudson2020} conducted two studies to construct and validate a measure of people's desire to change attachment characteristics followed by a 16-wave weekly longitudinal study totalling more than 4,000 participants combined.
In the first two studies, Hudson et al. found significant individual differences in desire to change attachment-related attributes. Crucially, these differences were related to measured trait levels of attachment anxiety and avoidance and to satisfaction with relevant life domains. This followed the theoretically expected pattern that people with greater dissatisfaction in relevant domains as well as people with higher measured levels of anxiety or avoidance were more likely to want to change those traits.
Finally, the 16–wave weekly longitudinal study showed that desire to change predicted observed change not only at the level of security-insecurity but in the expected domain such that people who wanted to become less anxious generally experienced less attachment anxiety over time and people wishing to decrease their avoidance generally did so. Impressively these effects remain significant when controlling for relationship status changes.


In summary, it appears that attachment is a relatively stable construct of individual differences which, apart from transient state-like changes following some major life events, changes and develops at a pace measured on the scale of years. Nevertheless, it is not clear exactly how or why changes to attachment style occur, and McConnel \cite{McConnell2011} points out that the factors influencing attachment in adulthood are complex, ranging from internal and behavioural factors such as coping and well-being to external factors like life events and environmental stress. This may make attachment more malleable through deliberately intervention in adulthood and more difficult to predict over very long periods as is shown by e.g., Pinquart et al. \cite{Pinquart2013}.
Further, individual differences and psychological factors, including coping strategies \cite{Zhang2004} and appraisals \cite{Fraley2021}, are important in mediating the effects of life events. The importance of psychological factors provides some support for the notion of volitional changes as has been convincingly demonstrated by Hudson et al. \cite{Hudson2020}.
Based on the available evidence, one must conclude that attachment styles are somewhat stable but that they change naturally throughout the life cycle \cite{Rowe2005,Doherty2004,Fraley1997} and in response to both life events \cite{Fraley2021} and deliberate interventions \cite{Hudson2020}.

With the relative stability and some understanding of the changing of attachment styles established, I will now turn to the importance of attachment security in a wider context. As such, the next sections cover the predictive value of the theory and its constructs with regard to relationships, daily life, health, psychopathology, and mental health. This is not meant to be an exhaustive review but rather I aim to elucidate the remarkable extent to which attachment permeates pivotal facets of human experience. Consequently it will serve to illustrate the value of the construct as pliable, accessible, and informative variable for patients and clinicians alike.

\subsection{Relationships and Daily Life}
Mental health goes beyond the absence of psychiatric illness or mental distress. A great deal of influence on relationships and daily life is associated with attachment classifications. Some of this influence is briefly covered here.

Concerning romantic relationships, attachment insecurity is associated with lower levels of relationship evaluations even when controlling for important factors such as gender roles, romantic beliefs, and self-esteem \cite{Rodriguez2021, Jones1996}.
Further, in promoting relationship satisfaction and security for anxious and avoidant individuals in romantic relationships, different strategies employed by their partners have different effects depending on attachment style \cite{Overall2015}. The insights gained from research into regulation of attachment insecurity in romantic relationships has also been suggested as a guide on how clinicians should approach patients with different attachment patterns to help them distinguish the best strategies for treatment and building a strong therapeutic alliance, as is covered in section \ref{sec:Differentiable approaches}.

It is also clear that attachment styles influence how individuals process events and attachment cues in their relationships. This may lead to biased expectations and perceptions of their relationships and partners and eventually interfere with well-functioning relationship processes \cite{Collins2007, Collins2004, Hazan1994, Mikulincer2003, Rodriguez2019}.
For instance, insecurely attached partners in romantic relationships tend to react in a counterproductive manner when faced with interdependence dilemmas in their romantic relations \cite{Simpson2012}.

Lastly, concerning everyday distress and social functioning, Sheinbaum et al. \cite{Sheinbaum2015} followed 206 young adults using an experience-sampling methodology.
They found that cognitive appraisals, social functioning, and moment-to-moment affective states were predicted by attachment style.
Namely, individuals with anxious attachment experienced higher negative affect, stress and perceived more social rejection in their day-to-day experience and interactions. Individuals with avoidant attachment, however, seemed to experience greater degree of deactivation such as decreased positive states and lowered desire for social interaction when alone.
Additionally, individuals with anxious attachment experienced social situations with low levels of rated "closeness" significantly more negatively than those with secure attachment \cite{Sheinbaum2015}.

\subsection{Health, Psychopathology, and Mental Health}
Attachment theory and its supporting evidence have serious implications for general healthcare with associations reaching from the symptoms experienced and reported to the tendency to visit health professionals.
In a survey-based study of 287 university students assessed twice ten weeks apart by Feeney and Ryan \cite{Feeney1994}, anxious attachment was linked to higher symptom-reporting. This association remained robust even when controlling for general negative emotionality, suggesting that those with anxious attachment may be more prone to experience or report signs of distress.
The same study also found that avoidant attachment was associated with fewer visits to health professionals, even when symptom reports were controlled for, suggesting that avoidant attachment is associated with delayed seeking of medical care \cite{Feeney1994}.

The occurrence and development of psychological distress may also be associated with attachment insecurity. A 2017 review by Dagan, Facompré, and Bernard \cite{Dagan2018} of 55 studies assessing a total of 4,386 participants using the AAI and well-validated depression measures found significant associations between attachment classification and depressive symptoms.
Namely, insecure attachment was associated with depressive symptoms. This effect seems to be carried primarily by the preoccupied participants as the dismissing participants alone did not exhibit significantly more depressive symptoms.

The notion that attachment insecurity is associated with clinically relevant psychological distress is supported by a large study of 5,645 participants by Meng, D'Arcy, and Adams \cite{Meng2015}.
In this study, the authors compared the use of various healthcare services across attachment classifications. They found that individuals who reported insecure attachment styles were significantly more likely to have used a variety of healthcare services including having had a session of psychotherapy or a prescription for pharmacological treatment of mental or behavioural problems. This was the case for both anxious and avoidant insecurity, while only individuals with anxious attachment were more likely to report having used online support groups or forums.

The phenomenology and symptomatology of psychopathology may also differ between individuals according to their differences on the attachment scales.
This was demonstrated by comprehensive study of 500 patients with psychosis from three countries by Korver-Nieberg et al. \cite{Korver-Nieberg2015} which found that attachment anxiety (but not avoidance) predicted the severity of both positive and affective symptoms while both anxiety and avoidance were associated with severity of hallucinations and persecution.
Similarly, in a smaller sample of 40 persons with serious psychopathology, Dozier \cite{Dozier1990} found that attachment security was associated with affective symptomology rather than thought disorders.

Among the most convincing evidence that attachment is associated with prevalence and phenomenology of mental distress and psychiatric illness is a 2009 review of 25 years of studies conducted with the AAI, totalling more than 10,000 interviews, by Bakermans-Kranenburg and van IJzendoorn \cite{Bakermanskranenburg2009}.
The authors found that across studies, anxious attachment was associated with predominantly internalising disorders such as borderline personality disorder, while avoidant attachment was associated with externalising disorders such as antisocial personality disorder.

Attachment orientation appears to influence therapy outcomes as well. As is described in section \ref{sec:Applying attachment in therapy}, this influence may be mediated by lower ratings of therapeutic alliance associated with insecure attachment \cite{Baier2020}, which in turn may be caused by difficulties with building and maintaining a strong alliance by resolving ruptures.
The importance of attachment for therapy outcomes is made apparent in a number of studies. Most notably, a review of three meta-analyses with a combined \textit{N} of 1,467 found a medium effect size of attachment anxiety on treatment outcome ($d=-.46$) while overall security had a smaller yet relevant effect size ($d=.37$) \cite{Levy2011}. This adds further credibility to the notion that greater attachment anxiety is associated with poorer post-treatment outcomes, while the inverse is true for attachment security. Avoidance was not significantly correlated with therapy outcomes.


\section{Applying Attachment Theory in Therapy}
\label{sec:Applying attachment in therapy}
Although the theory's two main founders, Bowlby and Ainsworth, were both clinicians, the theory would not begin seeing serious use as a framework for psychotherapy until the very end of the 20th century \cite{Slade2019}.
Bowlby was himself a psychoanalyst and psychotherapist and the therapeutic process and how it could be informed by attachment theory was one of the primary interests for Bowlby, and thus, an important driver for the formation of the theory.
Bowlby gave a number of lectures and published several articles giving explicit advice on the application of attachment theory for clinical practice \cite{Bowlby2005b}.
While pleased with the extended research into the development of children, on several occasions, Bowlby expressed his disappointment with how little momentum the theory had gathered in clinical circles \cite{Bowlby1988}.
In fact, Bowlby believed that the therapeutic relationship may be among the best in revealing attachment patterns \cite{Bowlby1988}

The specific ways in which attachment influences the process of therapy and the strategies that therapists employ to face the challenges of and leverage the opportunities offered by different attachment states of mind are complex.
There is however, a foundation of theory and research into this and related concepts, some of which is covered in the following sections on applying attachment-related concepts to clinical practice.
It should be noted that although the attachment security of therapists may be an important variable influencing the patient's experience and the therapist's effectiveness \cite{Mikulincer2013, Daniel2006, Dozier1994, Cologon2017, Talia2020}, it is not covered in this thesis.

The relevance of attachment, whether as a variable of individual difference to be accounted for or as in-session behaviours and dynamics to recognise and act on, is acknowledged by practitioners of almost all types of therapy today.
Across modalities, clinicians now view attachment dynamics as primary to the function of psychotherapy and many directly apply them to interpret their patients' experiences and to select and design interventions.
The importance of this is suported by Shorey and Snyder, who, in a 2006 review of the theoretical landscape \cite{Shorey2006}, conclude that an understanding of attachment theory is beneficial for the conceptualisation of client problems and in the selection and design of interventions.
They strongly suggest that an assessment of attachment style should be a standard part of treatment planning and argue that there is good reason to include attachment measurements not only in clinical work but also as an individual difference variable in psychotherapy and -pathology outcomes research \cite{Shorey2006}.

According to Slade and Holmes \cite{Slade2019}, attachment-informed therapy offers three core principles that apply universally, regardless of a patient's attachment style.
These principles remain valuable even before a formal assessment is conducted.

First, by recognising the basic attachment needs such as emotional safety and a secure base for exploration, therapists can more readily foster the environment and secure relationship needed for patients to address their challenges.

Second, the adaptation of therapeutic practice to specific needs and the therapist's ability to recognise attachment behaviours and reactions related to insecurity is a key competence characteristic of experienced practitioners. This clinical attention is relevant not only when the attachment style of the patient is known and has been assessed but for any patient as it provides valuable insight into in-session efforts of the patient to regulate their emotions.

Third, therapy can change attachment styles. In attachment-informed psychotherapy, the therapists aims to be a secure base for their clients to explore their experiences and emotions, thus fostering autonomy, openness, and an affectively coherent sense of self.
Through the therapeutic process, clients can become more effective at regulating their anxiety in relation to attachments, as in relation to anything, and can learn to gradually change their behaviours and patterns of thought, over time leading to measurable change in attachment styles.
These changes occur even in clinical samples with very serious psychopathology \cite{Fonagy1996}.
In fact, patients with borderline personality disorder may be among those who can expect the greatest shift in the direction of security \cite{Levy2006, Stovall2003}.

\subsection{The Therapeutic Alliance}
While the therapeutic alliance is not a construct directly under attachment theory, their relevance to each other are apparent. The therapeutic or working alliance is a well-studied measure highly related to therapeutic outcomes and experiences and the theoretical and empirical links between attachment and alliance constructs are well established.

\subsubsection*{What is the therapeutic alliance and why does it matter?}
Early conceptualisations of the working alliance, such as that by Edward Bordin \cite{Bordin1979}, emphasized an alignment between client and therapist regarding the goals and objectives of therapy. However, Bordin also recognized the importance of an emotional connection built on trust and mutual respect. He proposed that the working alliance is comprised of three key components: goal, task, and bond. The bond component refers to the attachment-like relationship that develops between therapist and client.

The therapeutic alliance has emerged as stand-out measure in the vast literature on and concepts surrounding psychotherapy. Summers and Barber \cite{Summers2003} argued in 2003 that this is because it is shown to have robust effects on treatment outcomes across therapist traditions and patient conditions.
A more recent review corroborates this stance as a meta-analysis of four decades of research on the therapeutic alliance, covering 295 studies with a total \textit{N} of more than 30,000 patients by Flückinger et al. \cite{Fluckinger2018} concluded that the positive correlation between various measures of both the therapeutic alliance and patient outcomes is very robust.
This was the case irrespective of treatment approach, patient characteristics and whether the therapy was conducted face-to-face or online.

The robust nature of these associations has garnered significant research attention within psychotherapy. This research has led some scholars to posit that a strong working alliance may be the central mechanism or primary facilitator of therapeutic change \cite{RodgersCailholBuiEtAl2010}.
In a 2020 review of 37 studies and the extent to which they meet criteria for mechanistic research, Baier, Kline, and Feeny \cite{Baier2020} found that the therapeutic alliance mediated therapeutic outcomes in 70 \% of the studies included.
Therefore, the therapeutic alliance remains a crucial variable for understanding therapeutic change, even in studies where it is not directly assessed.  This underscores the importance of investigating the mechanisms by which the alliance exerts its influence on treatment outcomes as well as the mechanisms by which the therapeutic alliance is built and maintained.

\subsubsection*{Patient attachment and therapeutic alliance}
The link between attachment and therapeutic relationship may seem intuitive and indeed there is evidence to suggest that the patient's experience of the therapeutic relationship is influenced by their AAI classification \cite{Talia2019}.
This is supported by a 2014 meta-analysis by Bernecker, Levy, and Ellison \cite{Bernecker2014}, where weak but very robust associations were found between patient-rated working alliance and both attachment anxiety and avoidance. Both relationships were negative, such that higher avoidance or anxiety predicted worse alliance ratings.
Patient attachment to their therapist is strongly related to ratings of the strength of therapeutic alliance. In fact, Mallinckrodt, Gantt, and Coble \cite{Mallinckrodt1995} found correlations of around .80 between the two measures. This suggests that they are not only strongly related but may, at least in some instances, be indications of the same underlying construct.

Attachment security may influence the therapeutic alliance in more than one way.
A 2011 meta-analysis by Diener and Monroe \cite{Diener2011} of the association between attachment style and self- and therapist-rated therapeutic alliance found that greater attachment security predicted stronger therapeutic alliances. This relationship was significant for both types of ratings but significantly stronger for patient ratings than therapist ratings of the alliance.
No significant differences in the strength of alliances were found between anxious and avoidant attachment styles.
This could suggest that the therapist's own attachment characteristics play an appreciable role in the building and assessment of the alliance and perhaps that therapists should place lower confidence in their own ratings of the alliance when the patient has greater attachment insecurity.

It could also be the case that the alliance building process differs depending on the patient's attachment characteristics.
In fact, dismissing patients seems to be less sensitive to or invested in the working relationship with their therapist overall. Following 36 former political prisoners over 10-12 months of therapy, Kanninnen, Salo, and Punamäki \cite{Kanninen2000} found that patients' view of the therapeutic alliance developed differently depending on their attachment style.
While all three groups (secure, dismissing, preoccupied) rated their alliance similarly after the first session, dismissing patients were the only group to not experience a drop followed by an increase towards the end of their treatment.
Rather, they rated their alliance similarly to the beginning in the middle of their treatment and then substantially worse at the end of treatment.
Related findings by Eames and Roth \cite{Eames2000}, who followed 30 patients, tracking the working alliance and ruptures to it as reported by both patients and therapists, may offer a partial explanation.
As the theory might predict, preoccupied attachment was associated with more reports of ruptures and dismissing attachment with fewer, perhaps indicating differing sensitivities to interpersonal issues.

It appears that dismissing patients are more stable in their evaluations of therapeutic relationships, but that they may require more reassurance or work specifically on the alliance later in the therapeutic process.
In contrast, preoccupied patients seem to hold more shifting and unstable views of the relationship with their therapist, requiring more work throughout by paying close attention to situations that could be interpreted as alliance ruptures.
This could suggest that therapists should invest in interventions targeting the therapeutic alliance at different times throughout the course of treatment depending on their patient's attachment style.

It is clear that both patients with insecure attachment and their therapists rate the therapeutic alliance lower than is the case for patients with secure attachment.
However, the development through time of the alliance appears to be differentiated as well. Preoccupied patients may experience more ruptures to the alliance while dismissing patients are less sensitive in this regard.
This is valuable and directly actionable insights for therapists, who may want to adjust their therapeutic approach and consider not just how much but \textit{when} to invest in building the alliance depending on their patients' displayed attachment characteristics.

\subsection{Differentiable Approaches}
\label{sec:Differentiable approaches}
Beyond the established influence of in-session attachment dynamics and patient classification on the therapeutic alliance, this paper explores a more transformative and potent application of attachment theory in psychotherapy.
This approach involves tailoring therapeutic practice to the specific needs and developmental potential associated with the patient's unique attachment characteristics and state of mind.
Although this approach is not formalised in one universally accepted manual, I will review research showcasing its potential and demonstrating that experienced and skilled therapists already employ this type of tailoring in their practice.

Part of the basis for this approach comes from the observation that patients behave very differently in therapy and that they do so in ways partly predictable from their attachment characteristics. This was shown in an exploratory study by Daniel \cite{Daniel2011} who analysed patients' speech patterns in both psychodynamic and cognitive-behavioural therapy.
Notable differences were that preoccupied patients talked more and had longer speech turns than their dismissing counterparts who tended to generate more pauses.
Looking into the narrative structure of patients' speech, Daniel found that preoccupied patients took more narrative initiative whereas dismissing patients were more passive.
Interestingly, these differences between the speech patterns of dismissing and preoccupied patients were present in both modalities of therapy. This is particularly interesting considering that the two approaches differ quite substantially in their focus, with the psychoanalytic approach emphasising relationships and cognitive-behavioural therapy working more directly with symptoms.
It also suggests, that attachment-informed tailoring of psychotherapy has the potential to transcend the specifics of therapeutic modality.

One method which arguably refines the analysis of in-session verbal behaviour in relation to patient attachment, is the Patient Attachment Coding System (PACS) developed by Talia, Daniels, and Miller-Bottome \cite{Talia2014, Talia2017}.
Analysing patterns in patients' in-session behaviour and management of attunement with their therapist, Talia and colleagues \cite{Talia2014} found significant differences according to patient attachment classification.
As expected, dismissing patients showed less contact-seeking behaviour than both secure and preoccupied patients, preferring to limit emotional proximity.
Somewhat surprisingly, secure patients also displayed significantly more contact-seeking behaviour than preoccupied patients, and, as expected, preoccupied patients resisted the help of therapists more than both secure and dismissing patients.

Following up, in a similar study of in-session verbal behaviour investigating the mending of ruptures in the therapeutic alliance, Miller-Bottome et al. \cite{MillerBottome2018} found that securely attached patients would more openly reveal their emotions and invite the therapist to acknowledge them to resolve the rupture.
Dismissing patients, however, would minimise their own contributions and attempt to limit disclosure of their own feelings, downplaying the importance or magnitude of both affect and implied alliance ruptures.
In contrast, preoccupied patients would minimise the contributions of the therapist, limiting the attending to and resolution of alliance ruptures.
This implies that therapists should not only differentiate their approach to building the therapeutic alliance but also their management and mending of ruptures according to the attachment characteristics of their clients. The importance of this tailoring to specific challenges posed by different attachment patterns is crucial because unresolved ruptures to the therapeutic alliance can prevent therapeutic progress.

This may partly explain the findings of Dozier \cite{Dozier1990}.
In this study of 40 patients with serious psychopathology, Dozier found that greater security predicted greater compliance with treatment.
On the contrary, greater avoidance was associated with rejecting treatment providers, less self-disclosure in therapy, and lower ratings of the use of treatment by providers.

How therapists adjust their practice according to their patients' attachment characteristics is the topic of some research.
In a study of 12 peer-nominated experienced and highly effective therapists, Daly and Mallinckrodt \cite{Daly2009} found that the therapists strategies for management of therapeutic distance and the relationship between themselves and the client were almost mirrored over the course of treatment for clients with high attachment anxiety and avoidance.
For clients high in anxiety, therapists would begin treatment with fulfilling clients' needs for emotional proximity and then gradually increase therapeutic distance, to let the client practise managing their frustration and eventually learn to function more autonomously.
For clients high in avoidance, the corrective attachment relationship developed in the opposite direction. Here, therapists would begin by maintaining a therapeutic distance that avoidant clients found comfortable before gradually increasing proximity to let clients become comfortable with emotional closeness.
These diverging strategies suggests that experienced and effective therapists, whether intuitively or deliberately, aim to create a type of corrective attachment relationship between themselves and their clients in order to facilitate favourable change in accordance with the client's attachment characteristics.

This approach is consistent with findings suggesting that matching clients and therapists according to their activating or deactivating approaches is beneficial.
Specifically, Tyrrell et al. \cite{Tyrrell1999} found that patients who were more deactivating with regard to attachment (i.e. had higher avoidance) had better therapeutic alliance ratings and functioned better with therapists who were less deactivating while the opposite was the case for patients who were less deactivating.

Findings from a study of client-nominated relationship-building incidents in early sessions of therapy confirm that clients with high avoidance prefer supportive interventions while secure clients prefer incidents with a high degree of exploration \cite{Janzen2010}.
Likewise, Petrowski et al. \cite{Petrowski2011} found that patients with high attachment anxiety found therapists with higher attachment avoidance more helpful, suggesting that the naturally deactivating tendencies of the avoidant therapists were preferred over the hyper-activating or more emotionally oriented approaches of more anxious therapists.

Arietta Slade \cite{Slade2016} concludes on the basis of much of this research that the work of skilled clinicians involves adapting treatment on the basis of their patients' characteristics. While neither this idea nor the specific techniques employed by experienced therapists are new, framing them in the context of attachment can guide inexperienced therapists and experts alike in more specific ways of adjusting to the needs of their patients, and even when in the therapeutic process to do so.
Namely, Slade concludes, in all cases, therapists should aim to build a strong alliance by responding "in style" with their patients early in the therapeutic process. Later, once the relationship is well established, carefully responding "out of style" can help patients develop the necessary capacities for overcoming their challenges in move beyond their mental distress. For therapists of preoccupied patients, the aim is to engage their patients emotionally, nudging them towards expression and engaging in relationships.
Conversely, the aim for therapists of preoccupied patients is to help them regulate their intense emotions, gain autonomy, and learn to find comfort in the closeness they seek.

\chapter{Present Study}
\section{On Automatic Classification}
This thesis attempts to assess the feasibility and utility of automatically classifying patients' attachment characteristics during psychotherapy.
A demonstration that language models can learn to classify attachment styles based on patient speech would, in my view, constitute an indication that the approach is feasible.
As such, and especially given the low amount data available in this project, a clinically applicable tool, ready for deployment and accurate to a clinically satisfactory level, is not the goal.
Rather, performance above the random baseline would indicate that there are generalisable patterns relevant to each of the three main attachment classifications and that these patterns can be picked up as signal by sufficiently complicated language models.
On the contrary, if statistical models of more than 100M parameters are unable to detect this signal at all, it is unlikely, at least at this stage of development in language modelling, that more data will improve performance.
This would indicate that the approach is, with the currently available technology, not worth pursuing over other potentially promising avenues of research.

Continuous - rather than categorical - measures of attachment may be considered both more theoretically congruent and empirically more appropriate and thus be preferred.
Future work should attend to this fact and work towards integrating it in the continued development of these types of methods.
This may be done by e.g. tagging individual PACS discursive markers in transcripts or developing models to directly score the subscales given measures from the PACS or AAI.

Because continuous measurements were not available at the development of this project, it aims instead to provide a proof of concept for the automatic classification of attachment using the categorical convention.
Nevertheless, the study of in-session behaviour remains so highly relevant that its primary counterarguments arguably lie in the feasibility and scalability.
The benefits to clinical practice, as outlined in section \ref{sec:Applying attachment in therapy}, are numerous. Of equal importance is the interest in and methods enabling investigation of in-session behaviour and its relevance to clinical outcomes which may enable further research that will transform the practice of psychotherapy and therapist training programmes.
Talia et al. \cite{Talia2017} point out that studies using the PACS can move beyond exploring the link between attachment and treatment outcomes and investigate associations between \textit{changes} in attachment and treatment outcomes.
Additionally, the notion of tracking in-session behaviour over time and associating it with therapist behaviours and interventions offers a promising and potentially impactful branch of research \cite{Slade2016, Talia2017}.
This thesis begins the work of addressing the feasibility challenge by automating the classification task.
In the future, this will enable scalable pipelines for research and clinical practice in which automatic speech recognition (a fast developing area in automated data processing) can feed into models such as the one presented here for classification or continuous assessment.

\subsection*{Machine and Clinical Judgement}
In the pursuit of automation and machine-driven decision-making, it is essential to consider the aspects of interpretability and fairness.
This is particularly crucial in the context of complex discursive behaviours, such as those studied in this project, where the interpretation of patient-therapist interactions often extends beyond the literal meaning of the words exchanged.
While traditional methods may struggle to capture the intricacies of these interactions, this thesis employs state-of-the-art natural language processing (NLP) techniques, acknowledging that they may not be fully interpretable.
Given the ultimate goal of developing a practical tool, the primary focus is on achieving optimal performance, even if it means sacrificing some degree of interpretability.
I argue that the end-users of such a tool, namely therapists and patients, would ultimately benefit more from a high-performing model than one that is easily interpretable.

The field of NLP was forever changed when the transformer architecture was made public in 2017 \cite{Vaswani2017}.
Since then, variants of the transformer architecture have represented the state-of-the-art in NLP.
While applications in the realm of clinical psychology have so far been limited, I posit that we now have the understanding, infrastructure, and data to begin leveraging the immense power inherent in this technology for improved treatment outcomes.
This seems to be in line with trends in the application of various NLP methods in fields related to clinical psychology.
For instance, in a 2023 review, Malgaroli et al. \cite{Malgaroli2023} report a rising trend in the publication of peer-reviewed literature applying NLP to mental health interventions research.
While not exclusively so, a large portion of the increase in publications is carried by an increase in applications of deep learning and word embedding techniques. More recently, Malgaroli et al. found that transformer models were the most applied NLP tool in the domain of mental health interventions, likely owing to their ability to extract semantically meaningful features, attention mechanisms, and relatively large context windows \cite{Malgaroli2023}.
Malgaroli et al. further report that transformer-based models outperform other input features on tasks such as identifying emotional valence and accurately rating symptoms.

A key feature which is particularly important for the study of abstract verbal behaviours and the management of attunement is the contextually influenced embeddings of words that is among the distinguishing features of the transformer model.
Namely, the attention mechanism allows the model to dynamically update its representation of words given their context.
This ability distinguishes the transformer model from more traditional methods in which words are necessarily assumed to have static meaning across contexts \cite{Demszky2023}.
Demzsky et al. \cite{Demszky2023} highlight the astonishing performance of modern language models and point to their efficiency and speed as major factors in scaling the analysis of psychologically relevant text data to a previously unthinkable degree.
I believe that this project is an example of this.

Clinical judgement is a complicated and subjective task and tends to produce far from perfect inter-rater agreement. For instance, Wagner et al. \cite{Wagner2013} found 76-80 \% inter-rater overlap for evaluations of mental health status when different experts rated the same samples in their dataset of psychotherapy text.
While this is generally considered good inter-rater agreement, Wagner et al. also point out its highlighting of the subjective nature of psychological evaluations.

Inter-rater agreement on psychiatric diagnosis tends to be quite high when patients are assessed through structured interviews. Ruskin et al. \cite{Ruskin1998} reported a mean $\kappa$ of .83 for major depression, bipolar disorder, panic disorder, and alcohol dependence when assessed using a structured clinical interview.
Psychotic disorders, however, tend to have substantially lower agreement between raters. For instance, Maj et al. \cite{Maj2000} found only $\kappa = .22$ for schizoaffective disorder, while manic and major depressive episodes produced $\kappa$ of .71 and .82, respectively.

In a study of 217 psychiatric consultations, Al-Huthail \cite{Al-Huthail2008} found that the accuracy of initial psychiatric impression by primary medical providers when compared with final psychiatric diagnosis was just 60 \% for cognitive disorders, 50 \% for depression and 46 \% for anxiety disorders.

Considering more directly the reliability of the gold standard attachment measure, the adult attachment interview (AAI) has produced remarkably high inter-rater agreement and test-retest reliability scores.
For instance, Sagi et al. \cite{Sagi1994} found 90-100 \% agreement between their raters on three-way classification (secure, anxious, avoidant) of their AAI transcripts ($\kappa = .82 - 1.0$ for comparisons between three raters). Their test-retest agreement across raters was 90 \% ($\kappa = .79$).
Similarly, Bakermans-Kranenburg and van IJzendoorn \cite{BakermansKranenburg1993} found test-retest reliability for three-way classification of the AAI of 78 \% ($\kappa = .63$).
More recently, Talia et al. \cite{Talia2020} found an inter-rater agreement of .88 on their test of a small sample of interviews.

The high reliability of structured psychiatric interviews as well as both the PACS and the AAI and the comparatively low reliability of unstructured clinical judgement suggests that conducting clinically relevant judgement of psychological phenomena in one's patients is an exceedingly difficult task.
These issues are not unrecognised by psychiatrists, and in fact Matuszak and Piasecki \cite{Matuszak2012} suggest that more structured approaches be applied to both the anamnesis and diagnosis processes.
Exacerbating the issue, it seems likely that in the somewhat unstructured psychotherapy dialogue in which the therapist's emotional and cognitive ressources are likely to be focused elsewhere, the task of accurately assessing a patient's psychological characteristics becomes increasingly difficult.
As such, rather than relying entirely on the intuitive judgement of therapists who may need to direct their focus elsewhere, I suggest that qualified decision support is warranted.
The union of machine and clinical judgement does not have to be a dichotomy.
In the ideal case, an automatic decision-support system for the assessment of attachment orientation can free up therapists' resources and allow them to be more present in their counselling sessions while providing more accurate and reliable repeated measurements to inform treatment and intervention design.

\section{Methods}

\subsection{Transformer Models}
The models applied in this project are all variants of the RoBERTa model first made available by Liu et al. in 2019 \cite{roberta}.
The RoBERTa name is derived from its claim of being a robustly optimised BERT approach.
As such it is fundamentally a BERT model and shares most of its architecture with its 2018 ancestor.

At their core, the BERT and RoBERTa architectures are transformer-based models.
The transformer is a unique type of artificial neural network originally demonstrated by Vaswani et al. in 2017 \cite{Vaswani2017} for NLP purposes.
The transformer embeds an input sequence by transforming it from a sequence of $n$ symbolic representations $(x_1, \ldots, x_n)$ to a sequence of continuous representation, $z = (z_1, \ldots, z_n)$, which may then be decoded to make decisions about a down-stream task such as classifying the input.
This is achieved by first tokenising the input sequence into recognised sub-words.
Each unique token is associated with a vocabulary ID, allowing the model to look it up in its vocabulary.
The vocabulary consists of learned vector representations of all tokens known to the model.
The vocabulary size for RoBERTa models is approximately 50,000 tokens \cite{roberta}.
Each vector represents a given token by its location in a specified dimension-space known as the embedding size.
The values of each vocabulary vector is learned during training, so that any given embedding theoretically encodes the semantic meaning of a token in relation to other tokens.
Larger embeddings enable richer representation but comes at the cost of greater memory and computational requirements.
The embedding dimension for all BERT models, including the base RoBERTa is 768 while the large version of RoBERTa uses richer representations with dimension size 1024 \cite{BERT,roberta}.

Once the input sequence begins passing through the model, most computation is parallelised, rather than completed sequentially as was the case with previous architectures.
This is what makes training transformer models efficient and allows state-of-the-art models to process large corpora of text in their training.
To maintain the important information carried by word order, the embeddings are infused with positional information through positional encoding.
This means that, following embedding and positional encoding, the model can process each token, keeping a semantic understanding of it based on its usage in the training corpus as well as knowledge of where in the sequence the token is \cite{Vaswani2017}.

What makes transformer models so powerful, however, is not only the rich embedding of positional and semantic information, but also its attention to context.
This attention mechanism allows the model to dynamically adapt its representation of tokens given their context, effectively letting each token influence all other tokens, but ensuring they only do so when relevant.
This is what allows transformer models to differentiate the financial institution from the bank of a river despite the two words having the same vocabulary ID.
More importantly, it allows the model to gain an abstract understanding of language beyond a static dictionary meaning of each word \cite{Vaswani2017}.

In transformers, attention is computed by taking the vector representation of the input sequence through three separate learned linear transformations ($W^q, W^k, W^v$), producing what are referred to as query ($q$), key ($k$), and value ($v$) reduced-dimension projections of the original embedding vectors.
The $W^q$, $W^k$, and $W^v$ weight matrices have dimension $d_e \times d_k$ where $d_e$ is the size of the input and $d_k$ is the dimension of the space the input is projected to.
In BERT-style models, including RoBERTa, $d_k = d_{model}/h$ where $d_{model}$ is the dimension of the hidden layers in the model, which is the same as the dimension of the original embeddings, 768 for base models and 1024 for the large variant of RoBERTa.
For all sizes of BERT and RoBERTa models this results in a sub-dimensional space of size 64.
Each token in any given sequence of length $n$ tokens is thus projected to three representations of size $d_k$ and the sequence is represented as three different matrices of shape $n \times d_k$.
Taking the dot-product of the query and key matrices and applying a softmax function to the result produces an attention mask which, when applied to the value matrix, weights all parts of the input according to how they should 'attend to' one another.
Conventionally, the dot-product of the query and key vectors is scaled by some magnitude, typically the square root of the dimension of the key vector, to avoid vanishing or exploding numbers.
Thus, the attention-mechanism in transformer models can be succinctly expressed as

$$Attention(Q,K,V) = softmax(\frac{QK^\intercal}{\sqrt{d_k}})V$$
In most transformer models, this process occurs multiple times in parallel with different learned transformations ($W_1^q, \ldots, W_h^q, W_1^k, \ldots, W_h^k, W_1^v, \ldots, W_h^v$) corresponding to $h$ attention heads, the outputs of which are eventually concatenated into a single output and passed to a simple linear layer.
This is referred to as multi-head attention and can be expressed as
$$MultiHead(Q, K, V) = Concat(head_1, ..., head_h)W^O$$
Where, $$head_i = Attention(Q W^Q_i, K W^K_i, V W^V_i)$$
and $W^O$ is the matrix of weights in the linear layer.

The output of any given multi-head transformer block (encoder) may be passed as input to another transformer block or to e.g. a classification head to determine the label of the original input sequence.
The base RoBERTa architecture consists of 12 encoders each with 12 attention heads while the large RoBERTa architecture employs 24 encoders each with 16 attention heads.

\subsubsection*{BERT}
BERT, published by Devlin et al. in 2018 \cite{BERT}, is one of the first transformer-based language models and its key innovation consists in using the masked language modelling objective and to train on full sentences, rather than the previously employed left-to-right or right-to-left strategy of predicting the next token or sequence based only on "previous" context.
It is from this novel approach that the model takes its name. The Bidirectional Encoder Representations from Transformers solves its objectives by encoding rich contextual embeddings using transformers.
BERT is trained on two objectives and takes two sequences as input during its training.
The first task is a masked language model objective where tokens are randomly masked, and the model is tasked with predicting its original vocabulary ID using context from both sides of the masked token. 
The second objective used when training BERT is a next-sequence prediction task, in which the model predicts whether the second sequence follows the first in the source text.
The model is trained to perform on both objectives using the same encoder weights.
According to Devlin et al., this setup forces the model to learn meaningful embeddings and to consider both the meaning of masked tokens and how it is affected by context on either side of it.
The original BERT was trained on two datasets: BookCorpus, containing 800M words, and English Wikipedia, which was 2,500M words at the time, with a total weight of 13 or 16 GB depending on preprocessing \cite{BERT, roberta}.

\subsubsection*{RoBERTa}
The RoBERTa model primarily applied in this project develops on the architecture and training methodology proposed by Devlin et al. \cite{BERT} and trains on a larger corpus of text, adding data from online sources such as CommonCrawl to the original BERT pre-training data to reach a total of 160GB of raw text.
Liu et al. \cite{roberta} replicate most of the BERT methodology, but choose to train RoBERTa only on the masked token prediction task. They also make other changes such as increasing the size of batches during training, applying byte-level text encoding to create a larger and more flexible vocabulary, and including longer sequences than the original BERT had been exposed to.
The main difference in training methodology, however, is found in the data and its augmentation.
Namely, the Liu et al. showed that BERT was significantly under-trained by collecting a significantly larger corpus of text and by making each sentence in this corpus available to the model on several occasions.
To avoid the model overfitting to specific instances that it had previously encountered, the RoBERTa methodology applies dynamic masking such that each token is masked with probability .15 each time the model encounters a sequence.
This essentially means that any given sequence can be presented to the model as a new task on several occasions while limiting its opportunity to overfit.
According to Liu et al., the combination of training on more data with longer inputs and larger batches, using dynamic masking, and omitting the next-sequence prediction task produces a more robust model with more resilient and generalisable embeddings.
They demonstrate this through RoBERTa outperforming not only the original BERT but also all models published in the interim on all applied benchmarks \cite{roberta}.

\subsubsection*{FIne-Tuning}
The generalised understanding of language gained from masked language modelling means that RoBERTa models are well suited for fine-tuning to more specific tasks.
In this case, this is performed by adding a classification head to the model which takes as input the output of the final encoder layer. This classification head is a simple fully connected feed-forward neural network and can consist of any number of layers.
During training, the parameters of both the classification head and the encoder are updated to let the model gain specialised knowledge needed for the specific task.

\subsection{Data}
\subsubsection*{Task Data}
The labelled data for this project consists of the English part of the dataset from the 2017 validation study of the PACS \cite{Talia2017}.

\subsubsection*{Domain Data}


\subsection{Approach}
Lorem ipsum

\section{Results}

\section{Discussion}

\subsection{Implications}

\subsection{Limitations}

\subsection{Future Work}

\section{Conclusion}


\bibliography{references}

\appendix

\end{document}
