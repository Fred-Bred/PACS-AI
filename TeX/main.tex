\documentclass[12pt]{report}
\usepackage{cite}
\usepackage{amsmath,amssymb,amsfonts}
\usepackage{algorithmic}
\usepackage{graphicx}
\usepackage{textcomp}
\usepackage{xcolor}
\usepackage{booktabs}
\usepackage{tabularx}
\usepackage{hyperref}

%KUstyle
% \usepackage{KUstyle}
% \ptype{Social Data Science}
% \subtitle{An Attachment-Based Therapy Metric}


% Citation style packages
%\usepackage{apacite} % Package for APA citations
%\bibliographystyle{apacite}
\bibliographystyle{plain}

% This change the content of the frontpage
\title{Anxious or Avoidant? Securing Reliable Repeated Measures of Adult Attachment Through Machine Learning}
\author{Frederik Bredgaard}
\date{May 31st 2024}

\renewcommand{\contentsname}{Table of content}

\begin{document}

\maketitle
\
\tableofcontents

\chapter{Background and Theory}
\section{Introduction}
While the development of attachment theory has contiously delivered new and valuable insights into the development of children and the relationships, pathologies, and treatments of adults through the last six decades since its inception, its clinical application has remained mostly theoretical until somewhat recently. Arguably, this is not for a lack of clinical relevance but rather due, in large part, to the cumbersome measurement of the constructs belonging to the theory. As will be covered below, several instruments and methods for the assessment of attachment style exists. However, the gold standard for assessing adult attachment, the Adult Attachment Interview (AAI) \cite{AAITest}, is a time-consuming measure, making its application in large-scale research and clinical settings rather limited. The objective of this thesis is to develop on existing methods such as the Patient Attachment Coding System \cite{Talia2017}, adding a degree of automation through language modelling approaches derived from machine learning. While the available data is limited at this stage of the field, I believe that future work can build on the approach developed here to produce statistical models for autocoding the AAI as well as a clinically relevant tool that can assist clinicians and researchers alike by making the assessment of attachment in adults more easily available and significantly more scalable.

As such, I will investigate the feasibility and utility of automatically assessing psychotherapy patients' attachment characteristics.

The clinical utility of the approach is assessed first, through a review of the link between theory and empirical data, including implications on health, happiness, and well-being.

Second, the feasibility of automatically classifying a patient's attachment style is investigated through a series of language modelling experiments, based on the limited available data.


\section{Attachment Theory}
This section outlines the most central elements of attachment theory from its development to the clinical relevance of different patterns of attachment and how the theory more broadly explains or mediates the effects of psychotherapy.

From this theoretical perspective, an attachment may be understood as an affectional tie formed between an individual and some other and which is characteristed by behaviours seeking to gain and maintain proximity to the other. This tie and its associated behaviours bind the two individuals together in space and endures over time but the proximity-seeking behaviours are particularly prevalent during times of distress \cite{Ainsworth1970,Bowlby1988}.

\subsection{Theoretical Development}
The theory of attachment is a fundamentally ethological approach, which at the time of its development sought to explain behaviours that were poorly accounted for in existing theories. The initial development of the theory is often credited to John Bowlby, who originally formulated it from a psychoanalytic, Freudian perspective. However, the framework would soon develop through a more empirically informed approach, not least thanks to the groundbreaking work on infant attachment done by Mary Ainsworth and her colleagues \cite{Ainsworth1970}.

Bowlby's arrival at the tenets of attachment theory was influenced by the psychoanalytic exploration of object relations - a particular area of Freudian thought concerned with ego development and the relation of the psyche to external objects and persons. Bowlby, however, viewed the apparent attachment behaviours as being mostly adaptive to external needs and stimuli rather than as responses to internal fantasies. In line with the tradition that inspired him, Bowlby based his initial theories on case studies, particularly of delinquent children (e.g. \cite{bowlby1946thieves}). From these cases, Bowlby and his colleagues found that children displaying criminal or otherwise problematic behaviours or mental or emotional distress typically struggled to form meaningful relationships and many had been repeatedly institutionalised or moved between foster homes \cite{bowlby1951WHO}.

While the foundation for the theory and its conclusions had been laid long before, the first widely influential published piece on the effects of childhood care and attachment on mental health was Bowlby's 1951 monograph published by the World Health Organization under the title \textit{Maternal Care and Mental Health} \cite{bowlby1951WHO}. The central assertion of this work, which reviewed the limited existing empirical data, was that healthy infant development presumes a warm and continuous relationship with the mother. At the time, this proposition was controversial as it broke with traditional views on child rearing and development. Further, the supporting evidence was limited and neither psychology nor medicine offered any coherent well-developed theoretical explanations for the mechanisms leading from maternal care to mental health \cite{Bowlby1988, who1962deprivation}. Nevertheless, the publication sucessfully brought attention to the importance of maternal care for developing children.

Following the critique of his 1951 monograph, Bowlby 

\section{The Measurement of Attachment}
Lorem

\subsection{The Strange Situation}
Lorem

\subsection{Adult Attachment Interview}
The adult attachment interview is a semistrcutured interview that asks respondents about autobiographical memories from early childhood related to attachment. The interviewer asks respondents to review and evaluate these experiences from their current adult perspectives. The interview is coded according to the ways in which these childhood experiences are described and the reflections offered by respondents. Thus, the coding process is less concerned with the content of the specific content of memories and more interested in the organisation of thoughts and memories. This cognitive organisation according to early relationships is categorised into four attachment styles roughly corresponding to those observed in children and in most other measures of attachment patterns. These styles are labelled secure, dismissing, preoccupied, and disorganised \cite{Hesse1999, AAITest}.

The measurement of attachment in adults is typically made using the de facto gold standard of attachment research, the Adult Attachment Interview (AAI) \cite{AAITest, Talia2019, haltigan2014adult}. The administration of the AAI takes between 60 and 90 minutes while the required verbatim transcription may take four to ten hours and coding of the transcript is expexted to take at least four hours by a coder requiring extensive training, prompting some researcher to search for simpler or less ressource-intensive methods of measurement \cite{Haas1994}.

\subsection{Patient Attachment Coding System}
Lorem ipsum \cite{Talia2017}
\section{Attachment and Therapeutic Alliance}

\section{Therapist Attunement}

\section{Psychotherapy Research}

\chapter{Methods}
Lorem ipsum

\section{Data}
Lorem ipsum

\section{Approach}
Lorem ipsum

\bibliography{references}

\appendix
\chapter{Lorem ipsum}
Lorem ipsum

\chapter{Dolor}

\end{document}
