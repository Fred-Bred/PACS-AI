\documentclass[12pt]{report}
\usepackage{cite}
\usepackage{amsmath,amssymb,amsfonts}
\usepackage{algorithmic}
\usepackage{graphicx}
\usepackage{textcomp}
\usepackage{xcolor}
\usepackage{booktabs}
\usepackage{tabularx}
\usepackage{hyperref}

%KUstyle
% \usepackage{KUstyle}
% \ptype{Social Data Science}
% \subtitle{An Attachment-Based Therapy Metric}


% Citation style packages
%\usepackage{apacite} % Package for APA citations
%\bibliographystyle{apacite}
\bibliographystyle{plain}

% This change the content of the frontpage
\title{Anxious or Avoidant? Securing Reliable Repeated Measures of Adult Attachment Using Machine Learning}
\author{Frederik Bredgaard}
\date{May 31st 2024}

\renewcommand{\contentsname}{Table of content}

\begin{document}

\maketitle
\
\tableofcontents

\chapter{Background and Theory}


\section{Attachment Theory}
This section outlines the most central elements of attachment theory from its development to the clinical relevance of different patterns of attachment and how the theory more broadly explains or mediates the effects of psychotherapy.

From this theoretical perspective, an attachment may be understood as an affectional tie formed between an individual and some other and which is characteristed by behaviours seeking to gain and maintain proximity to the other \cite{Ainsworth1970}. This tie and its associated behaviours bind the two individuals together in space in what is commonly referred to as a dyad (e.g. \cite{Rodriguez2021, Overall2015,Bowlby2005}).
% (e.g., \cite{Bowlby2005, Rodriguez2021, Overall2015}).

\subsection{Theoretical Development}
Lorem ipsum

\section{The Measurement of Attachment}
Lorem

\subsection{The Strange Situation}
Lorem

\subsection{Adult Attachment Interview}
Lorem ipsum

\subsection{Patient Attachment Coding System}
Lorem ipsum \cite{Talia2017}
\section{Attachment and Therapeutic Alliance}

\section{Therapist Attunement}

\section{Psychotherapy Research}

\chapter{Methods}
Lorem ipsum

\section{Data}
Lorem ipsum

\section{Approach}
Lorem ipsum

\bibliography{references}

\appendix
\chapter{Lorem ipsum}
Lorem ipsum

\chapter{Dolor}

\end{document}
