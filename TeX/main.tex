\documentclass[12pt]{report}
\usepackage{cite}
\usepackage{amsmath,amssymb,amsfonts}
\usepackage{algorithmic}
\usepackage{graphicx}
\usepackage{textcomp}
\usepackage{xcolor}
\usepackage{booktabs}
\usepackage{tabularx}
\usepackage{hyperref}

%KUstyle
% \usepackage{KUstyle}
% \ptype{Social Data Science}
% \subtitle{An Attachment-Based Therapy Metric}


% Citation style packages
%\usepackage{apacite} % Package for APA citations
%\bibliographystyle{apacite}
\bibliographystyle{plain}

% This change the content of the frontpage
\title{Anxious or Avoidant? Securing Reliable Repeated Measures of Adult Attachment Through Machine Learning}
\author{Frederik Bredgaard}
\date{May 31st 2024}

\renewcommand{\contentsname}{Table of content}

\begin{document}

\maketitle
\
\tableofcontents

\chapter{Background and Theory}
\section{Introduction}
While the development of attachment theory has contiously delivered new and valuable insights into the development of children and the relationships, pathologies, and treatments of adults through the last six decades since its inception, its clinical application has remained mostly theoretical until somewhat recently. Arguably, this is not for a lack of clinical relevance but rather due, in large part, to the cumbersome measurement of the constructs belonging to the theory. As will be covered below, several instruments and methods for the assessment of attachment style exists. However, the gold standard for assessing adult attachment, the Adult Attachment Interview (AAI) \cite{AAITest}, is a time-consuming measure, making its application in large-scale research and clinical settings rather limited. The objective of this thesis is to develop on existing methods such as the Patient Attachment Coding System \cite{Talia2017}, adding a degree of automation through language modelling approaches derived from machine learning. While the available data is limited at this stage of the field, I believe that future work can build on the approach developed here to produce statistical models for autocoding the AAI as well as a clinically relevant tool that can assist clinicians and researchers alike by making the assessment of attachment in adults more easily available and significantly more scalable.

As such, I will investigate the feasibility and utility of automatically assessing psychotherapy patients' attachment characteristics.

The clinical utility of the approach is assessed first, through a review of the link between theory and empirical data, including implications on health, happiness, and well-being.

Second, the feasibility of automatically classifying a patient's attachment style is investigated through a series of language modelling experiments, based on the limited available data.


\section{Attachment Theory}
This section outlines the most central elements of attachment theory from its development to the clinical relevance of different patterns of attachment and how the theory more broadly explains or mediates the effects of psychotherapy.

From this theoretical perspective, an attachment may be understood as an affectional tie formed between an individual and some other and which is characteristed by behaviours seeking to gain and maintain proximity to the other. This tie and its associated behaviours bind the two individuals together in space and endures over time but the proximity-seeking behaviours are particularly prevalent during times of distress \cite{Ainsworth1970,Bowlby1988}.

\subsection{Theoretical Development}
The theory of attachment is a fundamentally ethological approach, which at the time of its development sought to explain behaviours that were poorly accounted for in existing theories. The initial development of the theory is often credited to John Bowlby, who originally formulated it from a psychoanalytic, Freudian perspective. However, the framework would soon develop through a more empirically informed approach, not least thanks to the groundbreaking work on infant attachment done by Mary Ainsworth and her colleagues \cite{Ainsworth1970}.

Bowlby's arrival at the tenets of attachment theory was influenced by the psychoanalytic exploration of object relations - a particular area of Freudian thought concerned with ego development and the relation of the psyche to external objects and persons. Bowlby, however, viewed the apparent attachment behaviours as being mostly adaptive to external needs and stimuli rather than as responses to internal fantasies. In line with the tradition that inspired him, Bowlby based his initial theories on case studies, particularly of delinquent children (e.g. \cite{bowlby1946thieves}). From these cases, Bowlby and his colleagues found that children displaying criminal or otherwise problematic behaviours or mental or emotional distress typically struggled to form meaningful relationships and many had been repeatedly institutionalised or moved between foster homes \cite{bowlby1951WHO}.

While the foundation for the theory and its conclusions had been laid long before, the first widely influential published piece on the effects of childhood care and attachment on mental health was Bowlby's 1951 monograph published by the World Health Organization under the title \textit{Maternal Care and Mental Health} \cite{bowlby1951WHO}. The central assertion of this work, which reviewed the limited existing empirical data, was that healthy infant development presumes a warm and continuous relationship with the mother. At the time, this proposition was controversial as it broke with traditional views on child rearing and development. Further, the supporting evidence was limited and neither psychology nor medicine offered any coherent well-developed theoretical explanations for the mechanisms leading from maternal care to mental health \cite{Bowlby1988, who1962deprivation}. Nevertheless, the publication sucessfully brought attention to the importance of maternal care for developing children.

Following the critique of his 1951 monograph, Bowlby 

\subsection{Current View of Attachment and Its Significance for Health and Daily Life}
While attachment is measured with a variety of methods with varying degrees of granularity today, the methods tend to agree on the underlying structure of attachment styles. This structure can be thought of as describing the main tenets of attachment theory as it is understood and applied today. This section offers a non-exhaustive review of relevant links between theory, practice, and observable outcomes. This relies on understanding the attachment styles measured using different approaches as essentially analogous or interchangeable even if they are presented with different names, as is custom in meta-analyses combining data from different measures (e.g., \cite{McConnell2011,Pinquart2013}).

\subsubsection{Stability and Change}
To understand predictive ability of attachment style, which constitutes the most direct link between theory and evidence, the assumption of stability must be addressed. Generally, an individual's state of mind with regards to attachment is theorised to be relatively stable throughout the lifespan. This theoretical assumption is based primarily on Bowlby's own speculation. Namely, in his influential book \textit{Attachment and Loss}, first published in 1969, Bowlby theorises that expectations regarding attachment figures are acquired in infancy and early childhood and thus should remain stable throughout adulthood \cite{bowlby1982attachment}. However, despite the stability of attachment styles being an axiom underpinning the theory as a whole, it has proven persistently difficult to confirm or reject the assertion empirically.

In the empirical literature, attachement classifications tend to show both stability and fluidity, with studies involving younger individuals finding only somewhat stable patterns with a greater degree of fluidity and studies of older individuals finding much greater stability.

Broadly, the available evidence points to mostly stable patterns of attachment over long periods of time. This is perhaps most convincingly demonstrated by a 2013 meta-analysis of 127 studies on attachment stability with a total \textit{N} of 21,072 by Pinquart, Fueßner, and Ahnert \cite{Pinquart2013}. Here, the authors found medium-sized stability with a test-retest coefficient of .39. However, while no significant stability was detectable in studies running longer than 15 years, coefficients were larger for shorter time intervals and for samples not comprised of at-risk children.
This suggests that the applied attachment measures have good reliability, and that while attachment displays good stability for shorter time intervals, life events, circumstances, and perhaps deliberate interventions can change it over longer intervals.

Similiarly, in a study by Zhang and Labouvie-Vief \cite{Zhang2004}, a sample of 370 individuals ranging in age form 15 to 87 was assessed regarding attachment, well-being, coping strategies, and depressive symptoms three times over a six year period. Attachment style was found to be reasonably stable although changes were observed. Test-rest reliability in this sample was between .40 and .49 between the first and second assessment and between .24 and .45 between the first and third measurement. This could indicate that attachment style does change, although this change is most likely to occur over a period of more than two years. For comparison, Zhang and Labouvie-Vief cite studies demonstrating that test-retest reliability of big five personality traits over a similar six year period tend to be around .60 - .80 after age 30 \cite{Costa1988,Roberts2000}.

The changes observed are themselves relevant to our understanding of the development, stability, and significance of attachment. Namely, Zhang and Labouvie-Vief found that change towards greater attachment insecurity was associated with defensive coping strategies characterised by rigid, immature and maladaptive ways of interacting with the world. Depressive symptoms was also a significant predictor of change towards greater insecurity. In contrast, change towards greater security was predicted by flexible and reality-oriented coping strategies and better perceived well-being.
Finally, Zhang and Labouvie-Vief found an effect of age effect on the direction of the observed changes. Specifically, it appears that over time, adults may become more secure and more dismissive but less preoccupied  \cite{Zhang2004}.

Lending further support to the notion of increased attachment stability with age, Consedine and Magai assessed attachement in 415 older adults at age 72 and again at age 78. In this sample, more than 80 \% of participants remained stable in their classification.

Studies into the hierarchies of attachment also further our understanding of the quality of the development of attachment through the lifespan.
Based on a study of mid- and late-adolescents recruited from a highschool and a university, respectively, Rowe and Carnelley \cite{Rowe2005} concluded that peers become increasingly important attachment figures as people age. While the undergraduate students assessed did not rate their parents as any less important to their core self than the highschoolers did, they considered their friends significantly more central to their core self.
This and other studies (e.g. \cite{Fraley1997,Doherty2004}) paint the picture of adolescents expanding their circle of close attachments to include their peers, but that this expansion does not come at the expense of the strength of attachment to parents or primary caregivers.
As adolescents become adults, they tend to show a decline in how highly they rate their parents as attachment figures and how much they rely on them for attachment-related needs. In their place, there is strong support for the notion that adults tend to rely more on friends and romantic partners as they age and in particular as their peer- and romantic relationships go longer \cite{Tancredy2006,Doherty2004,Fraley1997}.

It is generally accepted that changes to attachment style are common following major life events such as loss, change in relationship status, or becoming a parent. However, the specific causes of changes to attachment style are empirically not well understood. While Kirkpatrick and Hazan \cite{Kirkpatrick1994} found relationship initiation or break-up to be a mediator of attachment change, many studies are unable to attribute any observed changes directly to life events.
One such study is an 8-month examination of 144 young adults by Scharfe and Bartholomew \cite{Scharfe1994}. They used several forms of assessment and overall found moderate stability. The observed stability was noticeably higher for expert ratings than self-report measures, but, independent of measurement method, Scharfe and Bartholomew found no consistent relationship between changes in attachment security and life events between the two measurement occasions.
Similiarly, Cozzarelli et al. \cite{Cozzarelli2003} did not find strong associations between a long list of life events and attachment changes over a two-year period in a sample of 442 women who underwent an abortion.

The best evidence on how attachment changes has come more recently. In a 2021 study, Fraley, Gillath, and Deboeck followed a sample of over 4,000 people assessed in multiple waves for between six and 40 months. They found that changes in attachment security occurred following life-events related to relationships, career, family, and more. However, most changes were transient and the majority of people would revert to their original attachment style given enough time after most life events.
Nevertheless, some events did tend to lead to enduring changes, suggesting that some experiences are likely to affect change in attachment style.
The greatest enduring effects on increasing general attachment anxiety occurred following such events as entering retirement, receiving a work-related promotion, starting school or university, or moving to a new location. A shared characteristic of these events is a great change in social network. Following such life-changing events, individuals likely enter new communities, perhaps leaving others, and an increased anxiety around relationships may follow from spending more time with new, less close, relationships which inherently feel less secure. Simultaneously, spending less time in one's existing close relationships may weaken these relations or one's certainty of them,, effectively making them less secure.
Notably, Fraley et al. only found one event which had a significant lasting effect on \textit{decreasing} general attachment anxiety. This event was finding out that oneself or one's partner was pregnant, which had the second-largest enduring effect on general attachment anxiety, only exceeded in magnitude by the opposite effect of retiring from work.
Compared to the anxiety measure, avoidance was more stable in this sample. In fact, Fraley et al. only found two events to be associated with significant enduring changes in general attachment avoidance. These were getting married, which was associated with increased avoidance, and oneself or one's partner giving birth, which was associated with decreased avoidance.
Importantly, there were significant individual differences in the extent to which people changed. In line with previous research (e.g. \cite{Zhang2004}), positive or negative appraisals of the given experience were related to the extent of attachment changes on the individual level.

Finally, the concept of volitional change to one's attachment style is important, yet poorly described empirically. Similiarly to the research on the impact of life events, we have only recently seen strong direct evidence of deliberate changes to attachment security. This came in 2020 when Hudson, Chopik, and Briley \cite{Hudson2020} conducted two studies to construct and validate a measure of people's desire to change attachment characteristics followed by a 16-wave weekly longitudinal study totalling more than 4,000 participants combined.
In the first two studies, Hudson et al. found significant individual differences in desire to change attachment-related attributes. Crucially, these differences were related to measured trait levels of attachment anxiety and avoidance and to satisfaction with relevant life domains. This followed the theoretically expected pattern that people with greater dissatisfaction in relevant domains as well as people with higher measured levels of anxiety or avoidance were more likely to want to change those traits.
Finally, the 16–wave weekly longitudinal study showed that desire to change predicted observed change not only at the level of security-insecurity but in the expected domain such that people who wanted to become less anxious generally experienced less attachment anxiety over time and people wishing to decrease their avoidance generally did so. Impressively these effects remain significant when controlling for relationship status changes.


In summary, it appears that attachment is a relatively stable construct of individual differences which, apart from transient state-like changes following some major life events, changes and develops at a pace measured on the scale of years. Nevertheless, it is not clear exactly how or why changes to attachment style occur, and McConnel \cite{McConnell2011} points out that the factors influencing attachment in adulthood are complex, ranging from internal and behavioural factors such as coping and well-being to external factors like life events and environmental stress. This may make attachment more malleable through deliberately intervention in adulthood and more difficult to predict over very long periods as is shown by e.g., Pinquart et al. \cite{Pinquart2013}.
Further, individual differences and psychological factors, including coping strategies \cite{Zhang2004} and appraisals \cite{Fraley2021}, are important in mediating the effects of life events. The importance of psychological factors provides some support for the notion of volitional changes as has been convincingly demonstrated by Hudson et al \cite{Hudson2020}.
Based on the available evidence, one must conclude that attachment styles are somewhat stable but that they change naturally throughout the life cycle \cite{Rowe2005,Doherty2004,Fraley1997} and in response to both life events \cite{Fraley2021} and deliberate interventions \cite{Hudson2020}.

With the relative stability and some understanding of the changing of attachment styles established, I will now turn to the importance of attachment security in a wider context. As such, the next sections cover the predictive value of the theory and its constructs in regards to relationships, daily life, health, psychopathology, and mental health. This is not meant to be an exhaustive review but rather I aim to elucidate the remarkable extent to which attachment permeates pivotal facets of human experience. Consequently it will serve to illustrate the value of the construct as pliable, accessible, and informative variable for patients and clinicians alike.

\subsubsection{Relationships and Daily Life}

\subsubsection{Health, Psychopathology, and Mental Health}

\section{The Measurement of Attachment}
Lorem

\subsection{The Strange Situation}
Lorem

\subsection{Adult Attachment Interview}
The adult attachment interview is a semistructured interview that asks respondents about autobiographical memories from early childhood related to attachment. The interviewer asks respondents to review and evaluate these experiences from their current adult perspectives. The interview is coded according to the ways in which these childhood experiences are described and the reflections offered by respondents. Thus, the coding process is less concerned with the content of the specific content of memories and more interested in the organisation of thoughts and memories. This cognitive organisation according to early relationships is categorised into four attachment styles roughly corresponding to those observed in children and in most other measures of attachment patterns. These styles are labelled secure, dismissing, preoccupied, and disorganised \cite{Hesse1999, AAITest}.

The measurement of attachment in adults is typically made using the de facto gold standard of attachment research, the Adult Attachment Interview (AAI) \cite{AAITest, Talia2019, haltigan2014adult}. The administration of the AAI takes between 60 and 90 minutes while the required verbatim transcription may take four to ten hours and coding of the transcript is expexted to take at least four hours by a coder requiring extensive training, prompting some researcher to search for simpler or less ressource-intensive methods of measurement \cite{Haas1994}.

\subsection{Patient Attachment Coding System}
Lorem ipsum \cite{Talia2017}

\section{Applying Attachment Theory in Therapy}
The specific ways in which attachment influences the process of therapy and the strategies that therapists employ to face the challenges of and leverage the opportunities offered by different attachment states of mind are complex. There is however, a foundation of theory and research into this and related concepts, some of which is covered in the following sections on applying attachment-related concepts to clinical practice. It should be noted that although the attachment security of therapists may be an important variable influencing the patient's experience and the therapist's effectiveness\cite{Mikulincer2013, Daniel2006, Dozier1994, Cologon2017}, it is not covered in this thesis.
\subsection{The Therapeutic Alliance}

\subsection{Differentiable Approaches}

\section{Psychotherapy Research}

\chapter{Methods}
Lorem ipsum

\section{Data}
Lorem ipsum

\section{Approach}
Lorem ipsum

\bibliography{references}

\appendix
\chapter{Lorem ipsum}
Lorem ipsum

\chapter{Dolor}

\end{document}
